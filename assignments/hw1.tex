\documentclass[12pt]{exam}        %% What type of document you're writing.
\usepackage{graphicx}
\usepackage{times}
\usepackage[english]{babel} % English language/hyphenation
\usepackage{amsmath,amsfonts,amsthm,amssymb} % Math packages
%\usepackage{hyperref}
\usepackage{xcolor}
\usepackage[colorlinks = true,
            linkcolor = blue,
            urlcolor  = blue,
            citecolor = blue,
            anchorcolor = blue]{hyperref}

\usepackage{listings}
\usepackage{listings}
\usepackage{color} %red, green, blue, yellow, cyan, magenta, black, white
\definecolor{mygreen}{RGB}{28,172,0} % color values Red, Green, Blue
\definecolor{mylilas}{RGB}{170,55,241}





\begin{document}

\lstset{language=Matlab,%
    %basicstyle=\color{red},
    breaklines=true,%
    morekeywords={matlab2tikz},
    keywordstyle=\color{blue},%
    morekeywords=[2]{1}, keywordstyle=[2]{\color{black}},
    identifierstyle=\color{black},%
    stringstyle=\color{mylilas},
    commentstyle=\color{mygreen},%
    showstringspaces=false,%without this there will be a symbol in the places where there is a space
    numbers=left,%
    numberstyle={\tiny \color{black}},% size of the numbers
    numbersep=9pt, % this defines how far the numbers are from the text
    emph=[1]{for,end,break},emphstyle=[1]\color{blue}, %some words to emphasise
    %emph=[2]{word1,word2}, emphstyle=[2]{style},    
}



\title{\vspace{-0.5in}\textit{SMATH-358 Homework-1}, Due: Sept 30,Friday 11:59pm}
\date{\vspace{-60pt}}

\maketitle
\begin{center}
  \begin{tabular}[c]{ll}
    &Name: \underline{\phantom{----------------------------------------------------------}}
\end{tabular}
\end{center}



\begin{questions} 

\question  Show that the general solution of the logistic differential equation

\begin{equation*}
\begin{cases}
\dfrac{dP}{dt} = rP(1-P/K) \\[10pt]
P(0) = P_0
\end{cases}
\end{equation*}

can be written by $P(t) = \dfrac{KP_0e^{rt}}{K+P_0(e^{rt}-1)}$. 

\question Consider the following coupled linear ODE system;
\begin{equation*}
\begin{cases}
\dfrac{dx_1}{dt} = 4x_1 + 2x_2 +3e^t \\[10pt]
\dfrac{dx_2}{dt} = 2x_1 + x_2 +e^t
\end{cases}
\end{equation*} 
Verify that the functions $x_1=c_1+2c_2e^{5t}-\dfrac{e^{t}}{2}$ and $x_2=-2c_1+c_2e^{5t}-\dfrac{3e^{t}}{4}$ are the general solutions of this ODE system. Find a particular solution with the initial values $x_1(0)=1,x_2(0)=3$ .

\question In this question, you will replicate some of the results in \href{https://www.ncbi.nlm.nih.gov/pmc/articles/PMC4768423/pdf/12885_2016_Article_2164.pdf}{\color{blue}{\textit{Differences in predictions of ODE models of tumor growth: a cautionary example}}} by Murphy et.al. 
\begin{parts}
\part First, click on the link and read the paper. The paper concerns different tumor growth models and easy to read. 
\part In "Quantitative Example" section, they indicate that they used the data from \href{https://www.ncbi.nlm.nih.gov/pmc/articles/PMC2713268/pdf/1471-2164-10-301.pdf}{Systemic treatment of xenografts with vaccinia virus GLV-1h68 reveals the immunologic facet of oncolytic therapy} by Worschech et al by using \href{https://apps.automeris.io/wpd/}{WebPlotDigitizer}. Click on paper link, go to Figure 1A and extract the data from GI-101A control group(red points) using WebPlotDigitizer. You should learn how to use this tool, it is super easy. Download your data as a "csv" file and name it "tumor.csv". Make sure that the first column is the day and second column is the tumor size in your csv file. 
\part In Murphy et.al, replicate Figure-1 and Figure-2. Let's break down how we should do it;
\begin{enumerate}

\item There are 7 different growth models are discussed in the paper, namely, \textit{exponential, Mendelsohn, logistic, linear, surface, Gompertz, and Bertalanffy}. You will numerically solve each of them with matlab "ode45" command. To do so, we need a function to return these ODEs. Use the following template and complete the rest. Use the parameter values($a,b,c$ etc) given in Figure-1.
\begin{center}
\begin{lstlisting}
function dydt = GetODE(t,y,model)
    if model=='exponential'
        a = 0.0262;
        dydt = a*y;
    elseif model=='Mendelsohn'
        a = 0.286;
        b = 0.616;
        dydt = a*y^b;
    %complete rest of the code
    end
end
\end{lstlisting}
\end{center}

\item Just like we did in the class, define a step size $dt$, the last time point $T$ and form "tspan" etc and plot the solutions for the each model on the same figure. Figure out what $T$ should be to produce Figure-1.
\item Learn how to read a "csv" file and convert it to matrix. Read your "tumor.csv" you created and plot the first and second column just like in the paper. Notice that you should use \textbf{only the first 7 days} for this figure as indicated in the paper. Make sure all the plots you obtain in the previous item and these data points are on the same figure.
\item If you properly complete (1-3), you should be able to produce Figure-1. 
\item Change $T$ variable and and use the entire "tumor.csv" to produce Figure-2. 

\end{enumerate}

\item Generate SSR and $\text{AIC}_{C}$ values indicated in Figure-1. You can print them out with "fprintf" command. 
\item You should submit three files. First one is the pdf file for question \#1 and \#2, name it as [your first name].pdf. The second one is your  Matlab code as ".m" file and format it as "[your first name].m". The last one is  "tumor.csv" file. Make sure that when you run your matlab file, it correctly produces Figure-1 and item (d). I will not debug your code.

\item You should need help to finish your homework, please visit me during my office hours. You have two weeks to finish and there is no deadline, cheers! 

\end{parts}



\end{questions}
\end{document}